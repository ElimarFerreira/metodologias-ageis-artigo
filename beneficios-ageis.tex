\section{Benefícios das Metodologias Ágeis}
As metodologias ágeis surgiram como uma resposta às limitações dos modelos tradicionais de desenvolvimento de software, especialmente diante de mudanças frequentes de requisitos. Segundo Pressman e Maxim \cite{pressman2016engenharia}, abordagens flexíveis tendem a reduzir riscos e aumentar a aderência do produto às necessidades reais do cliente. Em contextos caracterizados por prazos reduzidos, elevada complexidade técnica e constantes alterações nos requisitos, torna-se essencial priorizar a entrega contínua de valor, a colaboração entre os envolvidos e a capacidade de adaptação. Nesse cenário, as abordagens ágeis oferecem benefícios significativos tanto para equipes quanto para organizações.

Um dos principais benefícios das metodologias ágeis é a capacidade de responder rapidamente às mudanças. Diferentemente de modelos preditivos, nos quais alterações tardias tendem a gerar elevados custos, os métodos ágeis incorporam a mudança como parte natural do processo, permitindo ajustes frequentes ao longo do ciclo de desenvolvimento. Isso resulta em produtos mais alinhados às reais necessidades dos usuários e do negócio. Conforme destacado pelo Manifesto Ágil \cite{beck2001manifesto}, a colaboração e a adaptação contínua são elementos fundamentais no desenvolvimento de software.

Outro aspecto relevante refere-se ao aumento da transparência e da comunicação. Práticas como reuniões diárias, revisões de Sprint e retrospectivas promovem o alinhamento contínuo entre os membros da equipe e os stakeholders. Essa comunicação frequente reduz ambiguidades, facilita a identificação precoce de problemas e contribui para a construção de um ambiente colaborativo e de confiança.

As metodologias ágeis também favorecem a melhoria contínua. Ao final de cada iteração, as equipes são incentivadas a refletir sobre seus processos e resultados, identificando oportunidades de aprimoramento. Esse ciclo sistemático de inspeção e adaptação contribui para o aumento gradual da eficiência, da qualidade do software e da maturidade da equipe ao longo do tempo.

Do ponto de vista organizacional, a adoção de métodos ágeis pode resultar em maior previsibilidade e redução de riscos. A entrega incremental permite que partes funcionais do software sejam disponibilizadas em intervalos regulares, possibilitando feedback antecipado e correções de rumo antes que problemas se tornem críticos. Focar em entregas pequenas e frequentes facilita o monitoramento do progresso do projeto.

O Manifesto Ágil também enfatiza a importância das pessoas e das interações em detrimento de processos excessivamente rígidos \cite{beck2001manifesto}. Esse enfoque humano é reforçado por Highsmith \cite{highsmith2001agile}, que destaca a colaboração contínua como fator essencial para o sucesso de projetos ágeis.

Estudos empíricos apontam benefícios concretos decorrentes da adoção de métodos ágeis, como o aumento da qualidade do software e da satisfação das equipes. A revisão sistemática conduzida por Dybå e Dingsøyr \cite{dyba2008empirical} indica que práticas ágeis contribuem positivamente para a gestão de projetos em ambientes complexos.

Por fim, as metodologias ágeis valorizam as pessoas e suas interações, reconhecendo que o sucesso de um projeto de software depende não apenas de processos e ferramentas, mas principalmente da colaboração entre indivíduos motivados e capacitados. Esse enfoque humano contribui para maior engajamento das equipes e para a construção de soluções mais eficazes e sustentáveis.


