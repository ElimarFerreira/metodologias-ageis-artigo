\begin{abstract}   
As metodologias ágeis têm se consolidado como uma abordagem eficaz para o desenvolvimento de software em ambientes caracterizados por constantes mudanças, elevada complexidade e necessidade de entregas rápidas. Este artigo apresenta uma análise sobre a importância das metodologias ágeis, com ênfase no framework Scrum, destacando seus principais conceitos, papéis, eventos e benefícios práticos. A partir de uma abordagem teórica fundamentada em literatura especializada, são discutidas as vantagens da adoção de métodos ágeis em comparação aos modelos tradicionais, evidenciando aspectos como flexibilidade, colaboração, melhoria contínua e redução de riscos. Além disso, o trabalho demonstra a aplicação de ferramentas e práticas de engenharia de software, como o uso do Scrum para gestão do projeto, o Git para versionamento de código e o LaTeX para a produção acadêmica do artigo. Os resultados apontam que as metodologias ágeis contribuem significativamente para a eficiência dos processos de desenvolvimento e para a entrega contínua de valor.
\end{abstract}

%
% Resumo na outra língua - se o texto for em Português, o abstract é em Inglês e vice-versa - 
% como parâmetros devem ser passadas as palavras-chave na outra língua, 
%separadas por ponto e vírgula, e grafadas em letras minúsculas, salvo 
%em nome próprio, abreviaturas, ..., em que podem ser usadas letras maiúsculas (ver exemplo abaixo).
%

\begin{englishabstract}{Agile Methodologies; Scrum; Software Engineering; Software Development.
}
Agile methodologies have become an effective approach to software development in environments characterized by constant changes, high complexity, and the need for rapid deliveries. This article presents an analysis of the importance of agile methodologies, with emphasis on the Scrum framework, highlighting its main concepts, roles, events, and practical benefits. Based on a theoretical approach supported by specialized literature, the advantages of adopting agile methods in comparison to traditional models are discussed, emphasizing aspects such as flexibility, collaboration, continuous improvement, and risk reduction. Additionally, this work demonstrates the application of software engineering tools and practices, including the use of Scrum for project management, Git for version control, and LaTeX for academic writing. The results indicate that agile methodologies significantly contribute to process efficiency and continuous value delivery.
\end{englishabstract}
