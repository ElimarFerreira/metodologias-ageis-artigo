O desenvolvimento de software tem passado por profundas transformações ao longo das últimas décadas, impulsionado pelo avanço tecnológico, pela crescente complexidade dos sistemas e pela necessidade de respostas rápidas às mudanças do mercado. Durante muitos anos, modelos tradicionais de desenvolvimento, como o modelo em cascata, dominaram a indústria, baseando-se em um planejamento rígido e em fases sequenciais bem definidas. A atividade é é considerada complexa, pois envolve múltiplos fatores técnicos, humanos e organizacionais. De acordo com Sommerville \cite{sommerville2011software}, a crescente complexidade dos sistemas e a constante evolução dos requisitos tornam desafiadora a adoção de modelos rígidos de desenvolvimento, especialmente em ambientes dinâmicos. No entanto, tais abordagens mostraram-se limitadas em contextos caracterizados por incertezas, mudanças frequentes de requisitos e forte interação com o cliente. 

Nesse cenário, surgiram as metodologias ágeis, que propõem uma mudança de paradigma ao priorizar a colaboração entre equipes, a entrega incremental de software funcional e a adaptação contínua ao longo do projeto. Fundamentadas nos princípios do Manifesto Ágil, essas metodologias buscam aumentar a eficiência do processo de desenvolvimento, reduzir desperdícios e gerar maior valor ao cliente.

Entre as abordagens ágeis mais difundidas, destaca-se o Scrum, um framework que organiza o trabalho em ciclos curtos e iterativos, denominados sprints, promovendo transparência, inspeção e adaptação contínua. Dessa forma, compreender a importância das metodologias ágeis e, em especial, do Scrum, torna-se essencial para profissionais de Tecnologia da Informação que atuam em ambientes dinâmicos e competitivos.

Assim, este trabalho tem como objetivo analisar a relevância das metodologias ágeis no desenvolvimento de software, abordando seus benefícios práticos, a estrutura do Scrum e sua aplicação como ferramenta de gestão de projetos, além de apresentar recursos técnicos utilizados na elaboração acadêmica do estudo.


%\obso{Aqui o orientador pode comentar.}
%\obsa{Aqui você (o orientando) comenta.}

%\section{Objetivos}
%\label{sec-objetivos}

%O objetivo deste trabalho é eu conseguir terminar este curso. Conforme expresso na \sectionautorefname~\ref{sec-historia}, vou usar \LaTeX\ para o texto ficar mais bonito e eu não precisar ficar trocando figuras de lugar.

%\section{Organização do trabalho}
%\label{sec-organizacao}

%No \chapterautorefname~\ref{cap-introducao} foram apresentadas...
%No \chapterautorefname~\ref{cap-metodologia}...

