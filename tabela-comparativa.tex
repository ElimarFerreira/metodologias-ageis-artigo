\section{Comparação entre Metodologias Tradicionais e Ágeis}

A escolha da metodologia de desenvolvimento de software influencia diretamente o planejamento, a execução e os resultados de um projeto. Enquanto os modelos tradicionais adotam uma abordagem mais linear e preditiva, as metodologias ágeis enfatizam a flexibilidade, a colaboração e a adaptação contínua. A Tabela~\ref{tab:comparacao-metodologias} apresenta uma comparação entre essas abordagens.

\begin{table}[H]
\centering
\caption{Comparação entre metodologias tradicionais e ágeis}
\label{tab:comparacao-metodologias}
\begin{tabular}{|p{4cm}|p{5cm}|p{5cm}|}
\hline
\textbf{Critério} & \textbf{Metodologias Tradicionais} & \textbf{Metodologias Ágeis} \\
\hline
Planejamento & Extenso e realizado no início do projeto & Iterativo e contínuo ao longo do projeto \\
\hline
Flexibilidade a mudanças & Baixa, mudanças são custosas & Alta, mudanças são bem-vindas \\
\hline
Entrega de valor & Ao final do projeto & Incremental e frequente \\
\hline
Participação do cliente & Limitada às fases iniciais e finais & Contínua durante o desenvolvimento \\
\hline
Gestão de riscos & Identificação tardia de problemas & Identificação precoce por meio de entregas frequentes \\
\hline
Foco principal & Processos e documentação & Pessoas, colaboração e software funcional \\
\hline
\end{tabular}
\end{table}
